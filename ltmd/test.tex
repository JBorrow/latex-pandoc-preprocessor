\section{Nuclear masses and binding energies}
%
%
%
%
%
%
Atoms are constituted of electron and the atomic nucleus. The nucleus is composed of protons and neutrons (they are referred to as \emph{nucleons}). The protons and neutrons have approximately the same mass. The proton is positively charged and the neutron has a neutral charge. 

The nucleons in the nuclei are bound together by a force much stronger than the repulsion created by the positive charges of the protons. This force, called strong force has been found to have the following characteristics: 
\begin{itemize}
\item It has a short range of the order of a few $\rm fm$.
\item It is attractive, except for a repulsive core for distances smaller than $\simeq 1\,\rm fm$. 
\end{itemize}  

The nuclei are characterised by
\begin{description} 
\item[Atomic number:] the number of protons in the nucleus. Because the number of proton is the same as the number of electrons in a neutral atom the atomic number specifies the chemical properties of the atom. It is normally denoted with $Z$   
\item[Mass number:] the number of proton and neutrons. As its name suggests it is roughly related to the mass of the nucleus in atomic mass units. It is usually denoted with $A$.
\end{description}

A nucleus with atomic number $Z$ and mass number $A$ has therefore $N=A-Z$ neutrons. Nuclides are represented as $\ce{^A X}$, $\ce{^{A}_{Z}X}$ or $\ce{^{A}_{Z}X_N}$ with $X$ the chemical symbol for the atom. 

The binding energy $B(A,Z)$ of a nucleus with atomic number $Z$ and mass number $Z$ is the difference between its atomic mass and the sum of the mass of its constituents. 
\[M(A,Z)=Z*M(\ce{^{1}H})+(A-Z)M(n)-B(A,Z)\]
(note that by using $m(\ce{^{1}H})$ instead of the proton mass the mass of the electrons are included automatically). This formula assumes that the energy difference coming from the electron binding energies is negligible (it is of the order of a few $\rm{eV}$). 

\begin{figure}
\includegraphics[scale=0.4]{BindingEnergyAll.png}
\includegraphics[scale=0.4]{BindingEnergyMostStable.png}
\caption{Binding energy per nucleon as a function of the mass number. The first figure shows all nuclides, while the second only shows the most stable isobar. The plain line corresponds to the semi-empirical mass formula eq. \ref{eq:SEMF} for odd $A$ nuclei.}\label{fig:bindingEnergies}
\end{figure}

The binding energies for all nuclei are shown in Fig.~\ref{fig:bindingEnergies}. The binding energies cannot be predicted accurately from first principle but an approximate formula can be postulated based on physical arguments, with parameters to be determined experimentally. 
\begin{equation}\label{eq:SEMF}
M(A,Z)=N M_n+Z M_p+Z\, m_e-a_V A + a_sA^{2/3}+a_c\frac{Z^2}{A^{1/3}}+a_a\frac{(N-Z)^2}{4A}+\frac{\delta}{A^{1/2}}
\end{equation}  
The first three term are the mass of the constituents of the atom. The remaining term can be justified from a physical argument, but their coefficient has to be determined experimentally. For all terms the nucleus is approximated as sphere whose radius is approximately proportional to $A^{1/3}$.
\begin{description}
\item[$-a_V A$: ] This is called the volume term. The strong force between nucleons is short-range, unlike the Coulomb force. The consequence is that nucleons only interact with other nucleons that are close enough, so that the contribution to the potential energy is the same for all nucleons and only depend on the density $\rho$ of nucleons in the nucleus. This density is more or less constant for nuclei from moderate to large $A$. The potential energy will be proportional to the number of nucleons and since the density is more or less constant it will be proportional to the volume.   
\item[$a_sA^{2/3}$:] In the above description we neglected the fact that the density of neighbours is smaller for nucleons at the boundary of the nucleus, these will not contribute as much as those in the middle of the nucleus. This term correct for this effect and is proportional to the area of the nucleus (and hence to $R^2=(A^{1/3})^2=A^{2/3}$).
\item[$a_c\frac{Z^2}{A^{1/3}}$:] This is the Coulomb term. The potential for each proton is proportional to the number of other protons so it will be proportional to $Z(Z-1)$. The Coulomb potential is proportional to $1/r$ so we expect a dependence proportional to $1/R=A^{-1/3}$, for large values of $Z$ we can replace $Z(Z-1)$ with $Z^2$.
\item[$a_a\frac{(N-Z)^2}{4A}$] This is the asymmetry term. Because nucleons have spin $1/2$ they obey Fermi statistics (no two identical nucleon can be in the same state). To illustrate how this term arises we can imagine that the neutrons and protons have the same energy levels, and they fill the lowest $n$ levels if there are $n$ nucleons of one type. If we start from a nucleus with equal number $n$ of protons and neutrons both type will have their $n$ lowest energy state filled. If we now try to replace a neutron with a proton, this cannot work straight-forwardly because the corresponding energy level is already occupied by a proton, so this new proton will have to go into a higher energy level, hence reducing the binding energy. The same happens if we try to exchange a proton with a neutron. A more careful calculation\footnote{see for example \cite{Lilley:2009zz}, p.39 and Appendices B and C} gives the $(Z-N)^2$ dependence but our simplistic argument explains why the asymmetry contributes to the mass of the nucleus.    
\item[$\frac{\delta}{A^{1/2}}$] This term is called the pairing term. As for electrons around a nucleus the nucleons also can lower their energy by grouping in pairs. The "best" scenario is when both the protons and the neutrons are present in an even number, the "worst" when both neutrons and protons are in an odd number and in the middle is the case where either the neutrons or protons are in an odd number while the other is even. So we expect different values of $\delta$ for each of these cases. The scaling as a function of $A$ is found to be as $A^{-1/2}$.     
\[\delta=\left\{\begin{array}{ccc}
-\delta_p& &\mbox{even-even nucleus}\\
0& &\mbox{odd-even nucleus}\\
+\delta_p& &\mbox{odd-odd nucleus}
\end{array}\right.\]
\end{description}

The formula in eq. \ref{eq:SEMF} is often referred to as the \emph{liquid drop model}. 
\section{Nuclear stability} 
Not all combinations of protons and neutrons result in a stable nucleus. There are a relatively small number of nuclides that can be observed, with a smaller that are stable. The stable nuclides are shown in Fig.~\ref{fig:stability}. 
\begin{figure}
\begin{center}
 \includegraphics[scale=0.7]{stability.pdf}
\end{center}
 \caption{Stability of the nuclides as a function of the number of protons $Z$and the number of neutrons $N$. Figure taken from \cite{nndc}.}\label{fig:stability}
\end{figure}
\subsection{Decay constants}
If a substance is made of atoms which have a probability $\lambda$ of decaying per time unit, the change of the number of atoms is given by
\[\frac{dN}{dt}=-\lambda N\;,\qquad\Rightarrow \quad N=N_0 e^{-\lambda t}\]
$\lambda$ is called the decay constant. If we want to obtain the mean life we consider the number of decays in a small time interval $dt$: $dN=\lambda N(t)\,dt$ these decays happen at time $t$ so in order to get the mean of the time before decay we need to weight this number of decays by the time $t$, to take all atoms into account we integrate over $t$ and normalise by the total number of atoms:
\[\tau\equiv<t>=\frac{1}{N_0}\int\limits_{0}^{N_0} t dN =\frac{1}{N_0}\int\limits_0^\infty t \lambda N_0 e^{-\lambda t}\,dt=-\lambda\int\limits_0^\infty\frac{1}{-\lambda}e^{-\lambda t}=\frac{1}{\lambda}\]
the mean time before decay is called the lifetime. Another time describing the decay is the so-called half time. It is defined as the time after which half of the elements of the sample will have decayed. It is different from the lifetime but is related through
\[e^{-\lambda t_{1/2}}=\frac12 \quad\Rightarrow \quad t_{1/2}=\frac{\ln(2)}{\lambda}=\tau \ln(2)\]
Figure \ref{fig:decayConstant} shows the exponential decay of the number of nuclei and the lifetime and half time.
\begin{figure}
\begin{center}
\includegraphics[scale=0.5]{decayConstant.png}
\caption{Fraction of atoms remaining as a function of time.}\label{fig:decayConstant}
\end{center}
\end{figure}
The frequency of decays in a material is called the \emph{activity}
\[A=-\frac{dN}{dt}=\lambda N\]
Commonly used units for the activity are Becquerel ($1 Bq=1\;\mbox{decay}/s$) or Curie ($1 Ci=3.7\cdot 10^{10}\;\rm Bq$). The activity of a sample decreases with time as the number of candidate nuclei for a decay diminishes.
\[A(t)=\lambda N(t)=\lambda N_0 e^{-\lambda t}\;.\]
To measure the lifetime of a substance one can either measure the time dependence of the activity or use the definition, the former only practical for relatively short-lived nuclei and the latter requiring a good knowledge of the number of nuclei of a given substance in the sample.

